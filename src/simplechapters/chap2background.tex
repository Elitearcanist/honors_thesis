\chapter{Background}
\label{ch:background}

Radar is a basic application of radio.
In a traditional radio system a transmitter emits electromagnetic waves which are then detected by a receiving device.
The receiving device can be quite distant from the transmitter allowing wireless communication across vast spaces. % unnecessary sentance
A radar system works by placing the transmitter and receiver a known distance apart, often very close.
The transmitter then sends a known signal, the receiver, rather than solely attempting to detect the main transmission,
detects the reflections of that signal off of physical objects in the signal's path.
By using precise timing and some simple math, the distance from the transmitter to the target object and back to the receiver can be calculated.
The simplest radar equation given in Eq. \ref{eq:radar_range} highlights this principle.
The distance to the target $R$ is based on the $\Delta t$ or the time the signal took to reach the target and come back,
times the speed of light over 2
(the 2 accounting for the two-way travel). $\Delta t$ must be very precisely determined,
or the distance will be off due to the magnitude of the speed of light.

\begin{equation} R = \frac{c_0 * \Delta t }{2} \label{eq:radar_range}\end{equation}

% An important aspect of radar is range resolution.
% Put simply a radar can only differentiate closely spaced objects based on its sample rate.
% % T o d o explain more about range resolution include a formula


The system implemented in this paper is a subset of radar called a frequency-modulated continuous-wave (FMCW) radar.
FMCW uses a frequency-modulated signal often called a chirp to simplify the signal processing.
A chirp is a transmitted signal that moves in frequency over time.
If the frequency is increased over time, this shape is called an up-chirp.
Chirps can be made in many ways, a common chirp configuration is a sawtooth shape (Eq. \ref{fig:fmcw}).
Frequency modulated signals have an added benefit that the time delay can
be calculated using the characteristics of the chirp.
In a pulse radar system the time delay from transmit to receive must be precisely measured.
With a frequency-modulated signal, the time delay is extracted from the frequency difference
between transmit and receive \cite{wolff_radartutorial_nodate}.
Using a frequency-modulated chirp the range equation can be written as in Eq. \ref{eq:chirp_range}.
$\Delta f$ represents the change in frequency from transmit to receive and $\frac{df}{dt}$
is the slope of the transmitted chirp.
$\Delta f$ is determined by subtracting the received signal from the transmitted one.
The resulting signal is the frequency difference from transmit to receive and can then be applied
to Eq. \ref{eq:chirp_range} to calculate distance.
The $f_D$ in Figure \ref{fig:fmcw} is the Doppler shift.
Doppler shift can be used to determine the velocity of the target.
However, Doppler shift is not used in this paper.

\begin{equation} R = \frac{c_0 * \Delta f }{2 * \frac{df}{dt}} \label{eq:chirp_range}\end{equation}

\begin{figure}[h]
    \includegraphics[width=0.5\textwidth, center]{fmcw_chirp.png}
    \caption{A series of sawtooth shaped up-chirps, red representing the transmitted chirp and green the delayed received signal.
    $\Delta f$ or $\Delta t$ can be used to calculate the range to the target using equations \ref{eq:radar_range}, \ref{eq:chirp_range}.}
    \label{fig:fmcw}
\end{figure}

\section{Beamforming}
\label{sec:beamforming}
% TODO needs some work, fill gaps, add pictures?
Beamforming is a powerful technique that can be applied to a radar system in analog or digital, the basic principle is the same in either case.
In a radio system with more than one receive antenna, often called an array, beamforming can be applied.
When an electromagnetic wave hits the receive elements, it's received at different times between elements.
When the received signals are added together, the phase offsets between channels will create a pattern of constructive and destructive interference, which creates a combined antenna pattern or an array pattern.
This effect can be used to steer the combined receive channels array pattern.
By applying a phase delay to the individual receive channels, the combined receiver array pattern can be manipulated to electronically point the direction the array is receiving from.
This technique has been effectively applied for decades to electronically control array patterns, allowing one-dimensional or two-dimensional beamforming.
Beamforming can be applied digitally.
Each receive channel is sampled through an analog-to-digital converter.
Then, with each channel's data, phase delay can be applied digitally to steer the array’s beam.
Digital beamforming has the added benefit that beams can be steered in software, meaning the data can be manipulated to form multiple beam directions simultaneously.


% SDRs
\section{Software Defined Radio}
\label{sec:software_defined_radio}
Software defined radios (SDRs) are a commercial product that has become immensely popular in the last few decades.
SDRs rely on advanced and configurable Radio Frequency (RF) front ends and Field Programmable Gate Arrays (FPGAs) for collecting and processing RF samples in real time.
RF front ends such as the Analog Devices AD9361 or Lime Microsystems LMS7002M have a configurable selection of frequencies and sample rates.
FPGA accelerates the signal processing allowing SDRs to be highly configurable.
They connect to computers using a USB or ethernet adapter. % TODO figure for SDR layout?
SDRs have become very versatile tool for a variety of RF applications.
Lower-end SDRs are very common for hobbyist projects and higher-end SDRs have found effective use in testing and substituting sophisticated RF systems.

