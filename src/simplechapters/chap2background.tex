\chapter{Background}
\label{ch:background}

Go over the knowledge required to understand what I have done
1. how radar works
- how FMCW radar works
2. SDRs


Radar is a basic application of radio communication.
In a traditional radio system a transmitter emits electromagnetic waves which can then be detected by a receiving device.
The receiving device can be quite distant from the transmitter allowing wireless communication across vast spaces. % unnecessary sentance
A radar system works by placing the transmitter and receiver a known distance apart, often very close for simplicity.
The transmitter then sends a known signal, the receiver, rather than solely attempting to detect the main transmission,
detects the reflections of that signal off of physical objects in the signal's path.
By using precise timing and some simple math, the distance from the receiver can be calculated.
The simplest radar equation given in \ref{eq:radar} highlights this principle.
The distance to the target $R$ is based on the $\Delta_t$ or the time the signal took to reach the target and come back,
times the speed of light over 2
(the 2 accounting for the there and back of $\Delta_t$). $\Delta_t$ must be very precise,
or the distance will be off by large distances due to the magnitude of the speed of light.

\begin{equation} R = \frac{c_0 * \Delta_t }{2} \label{eq:radar}\end{equation}


The system implemented in this paper is a subset of radar called a frequency-modulated continuous-wave (FMCW) radar.
FMCW uses a frequency-modulated signal often called a chirp to simplify the signal processing.
A chirp is a transmitted signal that moves up in frequency over time.

\begin{figure}[h]
    \includegraphics[width=0.5\textwidth, center]{fmcw_chirp.png}
    \caption{Caption}
    \label{fig:fmcw}
\end{figure}

\cite{wolff_radartutorial_nodate}


% SDRs
Software defined radios are a commercial product that has grown immensely in the last
