\chapter{Introduction}
\label{ch:intro}

In the 2024-2025 Winter Season, 22 skiers, snowboarders, snowmobilers, and hikers died due to avalanches \cite{noauthor_avalancheorg_nodate}. % TODO cite
Avalanches present a constant risk to those exploring the outdoors.

% \section{Definitions}
% \label{sec:definitions}

% Do I include a SDR section?

\section{Literature Review}
\label{sec:literature_review}

% Paragraph about SDR radar research
Software-defined radios have had a significant increase in usage in radar applications in the last decade \cite{feng_reviewing_2020-2}.
The increase in radar research using SDRs has come from the improvement in SDR capabilities.
Feng, Mughees and Wollesen's review of SDRs in radar systems is an effective summary of SDR radar research from 2010 to 2020.
My introductory review will focus on more recent publications in SDR Radar research
as well as work done at BYU in this research area.

More recent radar research with SDRs has seen improvements in processing through GPU acceleration \cite{li_design_2022}.
Another paper used a low end SDR, the "Adalm-PLUTO" to detect landmines for use in robotic landmine removal systems \cite{bossi_versatile_2022}.
This system functioned as a ground penetrating radar.
In 2024 a University in Bulgaria used SDRs as a frontend for a doppler radar,
they demonstrated that the radar could detect targets up to 50 meters and could detect distance and doppler shift \cite{ivanov_design_2024}.
Another notable paper is "A Phased Array Radar Based on Software-Defined Radio" \cite{chen_phased_2025}.
They did not use a commercial SDR, but the RF frontend used the AD9361 which is software defined.
The radar system in the paper is a custom SDR.
They demonstrated that their radar could use 8 channel beamforming to accurately beamform
and detect targets in a 90 degree scan angle.


\subsection{Prior BYU Work}
\label{sec:prior_byu_work}
SDR-based radar-systems have also been researched at BYU for several years.
One paper by Andrew Monk investigated the logistics and feasibility of an SDR as a Radar Front-End \cite{monk_exploration_2020}.
The paper focused on different software and hardware options for running an SDR-based Radar Front-End.
GNU Radio Companion, the LimeSuite GUI and API, and SoapySDR were considered as software options.
For hardware, the focus was on the LimeSDR-Mini as it was determined to be optimal by the paper's criteria
which was focused on a low-cost CubeSAT application. % TODO define CubeSAT?
Nicholas Kohls' paper from 2021 demonstrated that the LimeSDR-Mini
functioned as a short-range radar \cite{kohls_software_2021}.
It was shown that the LimeSDR-Mini could accurately detect targets at short distances.
The paper also noted the many shortcomings of the LimeSDR-Mini, such as
the high data corruption rate (about 25\%) and lack of phase coherency.
